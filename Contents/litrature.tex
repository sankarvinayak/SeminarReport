\chapter{Literature Review}
Technical writing is writing or drafting technical communication used in
technical and occupational fields\cite{india}, such as computer hardware and
software\cite{rpi}, engineering, chemistry, aeronautics, robotics,
finance\cite{japan}, medical, consumer electronics, biotechnology, and
forestry. Technical writing encompasses the largest sub-field in technical
communication. See figure \ref{net2} that shows the autonomous systems in
Internet.

\begin{figure}[h!]
	\centering
	\includegraphics[width=0.9\linewidth]{ospf}
	\caption{Autonomous System Hierarchy}
	\label{net2}
\end{figure}

\section{section1}
\lipsum[2] % Please comment this line and type in the introduction chapter

\subsection{title 2}
\lipsum[3] % Please comment this line and type in the introduction chapter

\noindent The system is described by the equation \ref{sys_eq1} below. Here y is the ordinate and x is the abscissa , m is the slope and c a constant.

\begin{equation} \label{sys_eq1}
	y = mx + c
\end{equation}
\noindent Page centered and unnumbered multiple equations. The * symbol supresses equation numbering.
% Page centered and unnumbered equations
\begin{align*}
	2x - 5y & =  8   \\
	3x + 9y & =  -12
\end{align*}

\noindent Side by side figures can be created using this environment. See fig \ref{wave} below.
\begin{figure}[h!]
	\centering
	\begin{subfigure}[b]{0.4\textwidth}
		\includegraphics[width=\textwidth]{sinewave}
		\caption{Sine Wave}
		\label{fig:1}
	\end{subfigure}
	\hspace{20mm}
	\begin{subfigure}[b]{0.4\textwidth}
		\includegraphics[width=\linewidth]{cosine}
		\caption{Cosine Wave}
		\label{fig:2}
	\end{subfigure}
	\caption{The Sine and Cosine waves}
	\label{wave}
\end{figure}