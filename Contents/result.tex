%==================================introduction.tex=============================
\chapter{Results and Discussion}
% \lipsum[5-7] % Please comment this line and type in the results chapter
% This section presents the findings of the research, including any statistical
% analyses or other data that support the conclusions.

% This section interprets the results and places them in the context of the
% existing literature, highlighting the implications and contributions of the
% research.
\section{Growing process of memory layer}
During the initialization phase, there was no connection between neurons in the
memory layer, but during the structure formation phase, new connections were
grown in the memory layer of the network. This network could grow different
connections in the memory layer to memorize different patterns. When the
threshold was set to smaller values the connection was more sparse and the
memory could remember more information with more availability of free space.
\section{Recall process}
To check the recall process all images are used in the structure formation
phase. Then used the same images in both parameter training phase and pruning
phase The memory layer generated observed. There is different response based on
the four different kernels used show by four images.When an image is fed into
the network, different parts of the image provoke different firing responses.As
described previously an image is fed into the network and the firing sequence
in output layer decides the result using majority voting technique. The results
show that the network could recall the images it memorized When it was supplied
with an unseen but similar image the network could show some association
ability with the new data and the previously memorized data
% \begin{table}[h!]
% 	\centering
% 	\caption{test table}
% 	\vspace*{5pt}
% 	\begin{tabular}{|c|c|c|}
% 		\hline
% 		Sl. No & Item 1 & Itm 2 \\ \hline
% 		1      & 37     & 45    \\ \hline
% 		2      & 42     & 23    \\ \hline
% 		3      & 47     & 1     \\ \hline
% 		4      & 52     & -21   \\ \hline
% 		5      & 57     & -43   \\ \hline
% 		6      & 62     & -65   \\ \hline
% 		7      & 67     & -87   \\ \hline
% 		8      & 72     & -109  \\ \hline
% 		9      & 77     & -131  \\ \hline
% 		10     & 82     & -153  \\ \hline
% 	\end{tabular}
% \end{table}