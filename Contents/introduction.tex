%==================================introduction.tex=============================
\chapter{Introduction}%

% \addcontentsline{toc}{chapter}{Introduction}%
The ability to store and retrieve associations between different stimuli is a
fundamental component of many cognitive processes, including perception,
learning, and memory. Associative memory is a type of memory system that allows
for the storage and retrieval of relationships between different items in
memory. It is a key component of many artificial intelligence and machine
learning systems and has been extensively studied in both neuroscience and
computer science.

Spiking neural networks (SNNs) are a type of neural network that can simulate
the dynamics of individual neurons and synapses in the brain. They are effective for modelling a range of cognitive and sensory processing
tasks and have potential applications in a variety of fields, including
computational neuroscience and machine learning.

This report presents a study on the construction of an associative memory
system using a spiking neural network. We describe the architecture and
training procedure for our system and evaluate its performance on a variety of
associative memory tasks. We discuss the implications of our results for the
use of SNNs in implementing associative memory systems and highlight their
potential applications in computational neuroscience and machine learning.


\thispagestyle{plain}