%==================================introduction.tex=============================
\chapter{Introduction}%

% \addcontentsline{toc}{chapter}{Introduction}%
The ability to store and retrieve associations between different stimuli is a
fundamental component of many cognitive processes, including perception,
learning, and memory. Associative memory is a type of memory system that allows
for the storage and retrieval of information based on the relationships between
different items in memory. It is a key component of a good deal of artificial
intelligence and machine learning systems and has been extensively studied in
both neuroscience and computer science. According to research done by google
for fast contextual adaptation of speech\cite{google} associative memory system
using ANN are efficient in case of contextual adaptation of information.

Spiking neural networks is a kind of neural network which is capable of
simulating the dynamics of individual synapses and neurons in the human brain.
They have potential applications in many different domains, including
computational neuroscience and machine learning, etc. They are efficient for
simulating a variety of cognitive and sensory processing tasks of biological
neural networks.

This work presents a study on generating an associative memory system using a
spiking neural network. This study discusses the architecture and training
process of the system and evaluates its performance as an associative memory.
It also examines the potential applications of SNNs in implementing associative
memory systems and also in the fields of computational neuroscience and machine
learning.

\thispagestyle{plain}